%%%%%%%%%%%%%%%%%%%%%%%%%%%%%%%%%%%%%%%%%%%%%%%%%%%%%%%%%%%%%
%%%%%%%%%%%%%%%%% General document options
%%%%%%%%%%%%%%%%%%%%%%%%%%%%%%%%%%%%%%%%%%%%%%%%%%%%%%%%%%%%%

\documentclass[11pt,c]{beamer}
\usetheme{Xion}

%%%%%%%%%%%%%%%%%%%%%%%%%%%%%%%%%%%%%%%%%%%%%%%%%%%%%%%%%%%%%
%%%%%%%%%%%%%%%%% Usual LaTeX Packages
%%%%%%%%%%%%%%%%%%%%%%%%%%%%%%%%%%%%%%%%%%%%%%%%%%%%%%%%%%%%%

\usepackage[english]{babel}
\usepackage[utf8]{inputenc}
\usepackage[T1]{fontenc}
\usepackage{ragged2e}		% For justified alignment


%%%%%%%%%%%%%%%%%%%%%%%%%%%%%%%%%%%%%%%%%%%%%%%%%%%%%%%%%%%%%
%%%%%%%%%%%%%%%%% Talk and speaker info
%%%%%%%%%%%%%%%%%%%%%%%%%%%%%%%%%%%%%%%%%%%%%%%%%%%%%%%%%%%%%

\title{Xion --- A Clean And Simple Beamer Theme}
\author{Anatoly Arlashin}
\date{\footnotesize Last updated: Aug 2017}

%%%%%%%%%%%%%%%%%%%%%%%%%%%%%%%%%%%%%%%%%%%%%%%%%%%%%%%%%%%%%
%%%%%%%%%%%%%%%%% Main Content
%%%%%%%%%%%%%%%%%%%%%%%%%%%%%%%%%%%%%%%%%%%%%%%%%%%%%%%%%%%%%

\begin{document}

\begin{frame}[plain] % plain option removes header/footer/navigation/sidebar
	\titlepage
\end{frame}

\begin{frame}{Outline}
	\tableofcontents
\end{frame}

\section{Introduction}
	\begin{frame}{Why create my own theme?}
		\justifying	
		First, that's the best way to learn Beamer inner mechanics. Almost all Beamer tutorials that can be found online deal with changing some of the elements of existing themes inside the slides, as opposed to in a theme file. Q\&As on StackExchange help a little more, but good answers for in-depth questions are still hard to find.
		
		\bigskip
		Second, all Beamer themes that I've seen were somehow flawed (to my liking) --- too bright, too dark, too fancy, too many pictures, weird colors, asymetric geometry etc. 
		
		\bigskip
		When I found Zurich theme, it seemed almost perfect, just a few tweaks here and there. Or so it seemed at first \ldots
	\end{frame}

\section{Theme Elements}
\subsection{Colors}
	\begin{frame}{Defined colors}
		The theme defines the following colors (and then actually uses only a few):
		\medskip
		\begin{columns}[t, onlytextwidth]
			\begin{column}{0.3\textwidth}\centering
				\textcolor{lightscarlet}{lightscarlet}
				
				\textcolor{mediumscarlet}{mediumscarlet}
				
				\textcolor{darkscarlet}{darkscarlet}
				
				\textcolor{crimson}{crimson}
				
				\textcolor{lightchameleon}{lightchameleon}
				
				\textcolor{mediumchameleon}{mediumchameleon}
				
				\textcolor{darkchameleon}{darkchameleon}
				
				\textcolor{turtlegreen}{turtlegreen}
				
				
				
				
			\end{column}
			\begin{column}{0.3\textwidth}\centering
				\textcolor{lightskyblue}{lightskyblue}
				
				\textcolor{mediumskyblue}{mediumskyblue}
				
				\textcolor{darkskyblue}{darkskyblue}
				
				\textcolor{jeans}{jeans}
				
				\textcolor{lightplum}{lightplum}
				
				\textcolor{mediumplum}{mediumplum}
				
				\textcolor{darkplum}{darkplum}
				
				\textcolor{regal}{regal}
				
			\end{column}
			\begin{column}{0.3\textwidth}\centering
				\textcolor{lightaluminium}{lightaluminium}
				
				\textcolor{mediumaluminium}{mediumaluminium}
				
				\textcolor{darkaluminium}{darkaluminium}
				
				\textcolor{lightgray}{lightgray}
				
				\textcolor{ash}{ash}
				
				\textcolor{charcoal}{charcoal}
				
				\textcolor{darkcharcoal}{darkcharcoal}
			\end{column}
		\end{columns}
	\end{frame}

\subsection{Text Blocks}
	\begin{frame}{Block Party}
		\begin{block}{Block}
			Normal block. Colorless, neutral.
		\end{block}
		
		\begin{exampleblock}{Example Block}
			Example block. (Potential uses: list of pros, relaxing facts)
		\end{exampleblock}
		
		\begin{alertblock}{Alert Block}
			Alert block. (Potential uses: list of cons, impending doom)
		\end{alertblock}
	\end{frame}

\subsection{Lists}
	\begin{frame}{Numbered and itemized lists}
	\begin{enumerate}
		\item
			Numbered lists are themed up to three levels			
		\item
			The first level has numbers in squares
		\begin{enumerate}\justifying
			\item
				The second level has
			\item
				two numbers in bold
				\begin{enumerate}\justifying
					\item
						Levels below third are not themed					
					\item 
						and will be rendered using default template.
				\end{enumerate}			
		\end{enumerate}
	\end{enumerate}
	\begin{itemize}
		\item
			Itemized lists are themed up to three levels
		\item
			The first level has solid squares
		\begin{itemize}
			\item
				The second level has solid triangles
			\item
				The third level has empty circles
				\begin{itemize}
					\item
					Levels below third are not themed
					\item
					and will be rendered using default template.	
				\end{itemize}	
		\end{itemize}
	\end{itemize}
	
	\end{frame}

\section{References}
	\begin{frame}{References}
		\begin{itemize} \justifying
			\item
				Theme/color combinations for built-in Beamer themes: \href{https://mpetroff.net/files/beamer-theme-matrix/}{one} and \href{http://deic.uab.es/~iblanes/beamer_gallery/}{two}
			\item
				A list of custom Beamer themes: \href{https://latex.simon04.net/}{one} and \href{https://github.com/martinbjeldbak/ultimate-beamer-theme-list}{two}
			\item
				Beamer manual: \href{http://tug.ctan.org/macros/latex/contrib/beamer/doc/beameruserguide.pdf}{CTAN} and \href{https://en.wikibooks.org/wiki/LaTeX/Presentations}{WikiBooks}
			\item
				\href{http://www.cpt.univ-mrs.fr/~masson/latex/Beamer-appearance-cheat-sheet.pdf}{Cheat-sheet} on Beamer commands and options
			\item
				A nice \href{https://tex.stackexchange.com/questions/146529/design-a-custom-beamer-theme-from-scratch}{example} of how to create a custom Beamer theme from scratch
			\item
				\href{https://github.com/ppletscher/beamerthemezurich}{Zurich} theme, which I modified to create this one
			\item
				\href{https://github.com/fliptanedo/FlipBeamerTheme}{Flip} thime, which is what Zurich theme was based upon
		\end{itemize}
	\end{frame}


\end{document}

